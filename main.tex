\documentclass[a4paper]{article}
\usepackage{fullpage}
\usepackage[T1]{fontenc}
\usepackage[utf8]{inputenc}
\usepackage{amsthm}
\usepackage{amsmath}
\usepackage{hyperref}
\usepackage{lmodern}
\usepackage{enumitem}

% For TikZ diagrams:
\usepackage{pgfplots}
\pgfplotsset{compat=1.11}

\newcommand{\BTC}{BTC}

\title{A Probabilistic Micropayment Scheme for Golem}
\author{Golem Team (\texttt{golem@imapp.pl})}

\newtheorem*{dfnt}{Definition}

\newtheorem*{exmp}{Example}
\setlength{\parskip}{.6em}

\begin{document}
\maketitle

\begin{abstract}
    We consider a setting where a payment is made by a single payer
    to a possibly large group of~recipients, each receiving only a small
    sum in the order of \$$0.01$. For such small sums transaction fees are
    relatively large even if we consider cryptocurrencies instead of
    bank-based transactions.

    Both payers and recipients are expected to repeatedly take part in
    many payments, but subsequent payments of a single payer may
    have different groups of recipients. Also, we
    assume that payments result from activities carried out in a
    decentralized peer-to-peer network, so we would like to avoid
    solutions relying on any trusted third party. We cannot therefore use
    existing solutions that require a central server or assume that a
    payer makes a series of micropayments to a single recipient (Bitcoin
    micropayment channels).

    Instead, we propose a probabilistic payment scheme in which only one
    recipient randomly chosen from a group of candidates is rewarded in
    a single payment. Our solution is based on Ethereum and is thus
    decentralized and avoids relying on a trusted third party. We describe
    an~Ethereum smart contract implementing a lottery used to reward
    recipients and calculate the cost of running it. Depending on the
    variant of the protocol, this cost is either proportional to the
    logarithm of the number of recipients or constant. In both cases,
    the cost of running a lottery with 1000 participants is similar
    to the cost of direct payments to 20 recipients.
\end{abstract}

\section{Introduction}

    This work is part of research conducted in the Golem project. The aim of Golem is to create a global prosumer
    market for computing power, in which producers may sell spare CPU time of their personal computers and consumers
    may acquire resources for computation-intensive tasks. In technical terms, Golem is implemented as a decentralized
    peer-to-peer network established by nodes running the Golem client software. For the purpose of this paper
    we assume that there are two types of nodes in the~Golem network: \textbf{requester nodes} that announce computing
    tasks and \textbf{compute nodes} that perform computations (in the actual implementation nodes may switch between
    both roles).
    A requester node partitions a task into multiple subtasks, specifies payment for each subtask and recruits
    a group of compute nodes, each of which downloads and completes one subtask. This is again a simplifying
    assumption, since in~principle a single compute node could complete multiple subtasks.
    Results of each subtask are sent back to the requester node. After the requester node collects the results of all
    subtasks, it performs the~payment step\footnote{In this scenario we assume that the compute nodes do not fail
    and eventually deliver the results and the requester node is willing to perform the~payment step.
    The first assumption may be implemented for example by ensuring each subtask is computed by more than one node
    (in parallel or in sequence) and the second one may be enforced by a suitable reputation system for the nodes,
    but this is outside the scope of this paper.}. This last step is the main focus of the paper.

    In Golem we handle a variety of tasks with different sizes and decomposable to a different degree. In~particular:
    \begin{itemize}
        \item Task partitioning can range from 10 to 10000 subtasks.
        \item Task value can range from \$1 to \$10 000.
        \item Payment for a single subtask may be as low as \$0.01 or even \$0.001 for some types of tasks
    \end{itemize}

    The main requirement for any payment solution used in Golem is that it should be decentralized and not depend
    on any trusted authorities (banks, brokers).


    We should not expect that after a payment for a subtask is made, another payment with the same payer and recipient
    will occur in a short time. That is, it may not be possible to accumulate payments made by one payer and transfer
    the larger sum using a single transaction. On the other hand, we may consider methods that assume recurring
    payments to a single payee. That is, a compute node does not necessarily have to be paid a tiny sum for each
    subtask but instead may be paid larger sums once in a while. The obvious requirement is that in the long run
    the node's income should approach the sum of subtask values that this node has computed.

    Thus we are looking for a micropayments solution that handles payments as small as \$0.01 with transaction fees
    in the order of \$0.0001 per payment (this does not necessarily mean \$0.0001 per transaction which may group many
    payments). This rules out traditional methods such as bank-based transactions, credit card transactions or PayPal,
    since transaction fees they incur are in the order of \$0.1, plus a percentage of the transaction value
    (see e.g. \cite{FRS}).
    It seems more promising to look at available micropayment schemes based on digital currencies such
    as Bitcoin \cite{BITCOIN}.

\section{Micropayments schemes for cryptocurrencies}

    We first consider a naive solution in which the requester pays each node participating in the computation
    with a separate Bitcoin transaction. Although Bitcoin transaction fees are lower than for bank-based transactions,
    a transaction smaller than 0.01 \BTC{} ($\sim$ \$2.5 at the moment of this writing) requires a fee of
    0.0001 \BTC{} ($\sim$ \$0.025) to discourage "dust" transactions that bloat the Bitcoin blockchain \cite{BITFEE}.
    Such a fee may thus be be higher than the transaction value.

    A Bitcoin transaction may include many outputs, each to a different recipient, which seems a good match for
    one-to-many payments. We can therefore consider making a single Bitcoin transaction with a separate output
    for each of the participants. In this case, the flat fee of 0.00001 \BTC{} for "dust" transactions applies only
    once to the whole transaction, so this is not an issue. However, the size of the transaction in bytes grows with
    each output and Bitcoin also has fees for large transactions: the default charge is 0.0001 \BTC{} per 1000 bytes.
    Using the formula for approximating transaction size from http://bitcoinfees.com/\cite{BITFEE}:
    \begin{displaymath}
	    \text{\texttt{transaction\_size}} = 148 * \text{\texttt{number\_of\_inputs}} +
	    34 * \text{\texttt{number\_of\_outputs}} + 10.
	\end{displaymath}
    We may estimate that a transaction with 100 payees will cost at least \$0.1 and the cost will rise as the coins
    become fragmented and require a large number of inputs. However, as fees in Bitcoin are market-based rather than
    hard coded, miners may at some point start requiring higher fees for processing transactions \cite{KASKALOGLU}.
    According to some estimates (\cite{ANDRESEN}) miners should require a fee of
    at least 0.0032 \BTC{} ($\sim$ \$0.8) for including each 1000 bytes in a block), to compensate for the fact that
    a larger block is more likely to be orphaned. Finally, we should point out that a receiver of "dust" payments will
    bear the cost of spending the tiny coins she receives, due to higher fees for transactions with large number of
    inputs. This may discourage potential users from participating in Golem. For these reasons, Bitcoin transactions
    with multiple outputs may be suitable for tasks with a relatively high value (say above \$10) or small number
    of participants. For other scenarios a different payment scheme has to be used.

    For completeness we also mention Bitcoin micropayment channels (\cite{BITCOINJ}), but this solution does not fit
    our setting as it assumes that a payer makes a series of micropayments to one recipient.
    There is also a number of custom micropayment solutions such as Coinbase Tip \cite{COINTIP}(no longer active),
    or ChangeTip \cite{CHANGETIP} but they rely on a trusted third-party that processes transactions off-chain,
    which we definitely want to avoid in Golem.


\subsection{Probabilistic payments schemes}
    We next turn our attention to probabilistic payment schemes. The idea is that instead of paying \$0.01 directly,
    the payer issues a "lottery ticket" for a lottery with \$1 prize with a $1/100$ chance of winning. The expected
    value of such ticket is \$0.01. The advantage is that on average only one ticket in 100 will lead to an actual
    transaction. Such scheme is proposed e.g. in \cite{RIVEST} and \cite{WHEELER}.
    A possible implementation of this scheme may use cryptographic hash functions, for example SHA-3. First, both
    the payer and the recipient draw (or choose) their numbers $n_P$ and $n_R$, respectively.
    Both parties initially keep their numbers secret. Then the recipient reveals $hash(n_R)$. In the next step,
    the payer reveals $hash(n_P)$. Finally, both parties reveal their numbers (in any order) and use them to decide
    if the recipient receives the prize: the recipient wins if and only if $n_P = n_R \mod N$, where $1/N$ is
    the probability of winning. Note that even if the recipient learns $n_P$ before revealing $n_R$, she cannot change
    her choice of $n_R$ since she committed to the original choice by revealing $hash(n_R)$ and it is considered
    unfeasible to find another number with the same hash.

    Obviously, this probabilistic procedure does not guarantee that the Golem node is fairly remunerated if it only
    takes part in a small number of tasks. However, as the number of tasks increases, the node's real income from
    lottery rewards will approach the amount she would get being paid for each task. For the requester node the
    situation is similar, but the probabilistic effects may be more problematic: suppose a requester node just
    joined Golem and now has to pay for its first task to 100 other nodes with tickets giving each receiver
    1/100 chance of winning the total task value. Now, since each ticket is evaluated independently,
    there is a 37\% chance that the requester will not have to pay anything but also a substantial 26\% chance that
    at least two tickets win, and almost 2\% chance that at least 4 tickets win. Therefore the requester should be
    prepared to pay 4 times more than expected, which may discourage users from joining Golem.

    We can modify this scheme to make it more predictable for the requester by ensuring that among the tickets issued
    to reward the participants of a single task, exactly one ticket is winning. For example, if there are 100
    participants then each of them has 1/100 chance of winning, as before, but the requester has guarantee that
    task value will have to be paid only once. In other words, after the task is completed the requester will have
    to organize a lottery for the participants, in which exactly one participant wins. The drawback of this approach
    is that it requires a protocol that involves a large group of participants: the payer and all the payees.
    Such a protocol is more complex than in a one-to-one scenario where each lottery ticket is evaluated separately.

    In the rest of this paper we describe and compare several possible payment schemes (most of them briefly) in the
    context of Ethereum platform.

\subsection{Ethereum}
    We estimate the cost (in terms of transaction fees) of several payment schemes implemented in Ethereum:

    \begin{enumerate}
        \item \textbf{Direct transfer (with batching).}
        This is a naive approach in which the requesting node makes a transfer to each compute node participating
        in the task.
        Simple transfer costs 21 000 gas units for every user, which is about \$0.001 at the time of this writing.
        We also consider a variant that uses batching\cite{BUTERIN} to pack all payments in one transaction.
        In this case the cost of sending ether is 9000 gas units per node plus the cost of sending all data
        to the contract (min. 2000 gas per node). With current gas prices this method gives transaction fees
        below 5\% for payments of \$0.01 or more. To handle tasks with lower payments to a single recipient and to
        account for fluctuations of gas price we have to find a better method.
	    \item \textbf{Subaccounts.}
        Another approach is to create a smart contract that keeps an account balance for every Golem node.
        Every new user must register his account in the contract and put some ether in contract storage (with user
        address as the key and ether amount as the value). In the payment step, the requester sends a list of payees
        and amounts to the contract and the accounts of the payees are increased by the specified amounts.
        User can pay out only from his own account. In this scenario, the cost of a single transaction is 5000 gas
        units per recipient (for modifying the storage) plus the cost of sending all data to the contract plus the
        cost of additional integrity checks (user must send enough ether etc.).
	    \item \textbf{Lottery.}
        This is a probabilistic approach in which the requester organizes a lottery in which only one compute node
        wins. In this scenario the cost is generated mainly by the size of data that needs to be transferred to
        the contract. This approach is described in Section \ref{sec:lottery}.
    \end{enumerate}

    We also mention two additional payment schemes based on Ethereum but do not attempt to estimate their cost:

    \begin{enumerate}
	    \item \textbf{Micropayment channels}
        Micropayment channels in context of Ethereum are described in \cite{BUTERIN}. They allow the user to send
        transactions off-chain. Unfortunately, as in the case of Bitcoin micropayment channels \cite{BITCOINJ} this
        solution is designed for many 1-1 microtransactions between two given parties. The cost of opening a channel
        is quite large (at least the cost of adding 5 new storage fields 20 000 gas each) and channel endpoints might
        never cooperate again. Additionally, a micropayment channel requires an escrow, so if a requester has to open
        many channels the deposits may sum up to a large amount.
	\item \textbf{Bank-set approach.}
        Most micropayments protocols for P2P networks require some sort of a centralized institution (bank) that is
        in charge of approving transactions \cite{JAIN}. One idea is to replace this institution with a set of peers
        (an approach similar to the Karma protocol \cite{VISHNUMURTHY}) that keep track of transactions and user accounts.
        The requester pays the total amount of ether for each task into one joint Ethereum contract.
        If a user wants to pay out her share from her account, the nodes in the bank-set vote for or against the payout.
        Unfortunately it is not clear how to choose bank-set representatives, how Ethereum contract should know
        current representatives, how should they vote and how to design a reward/punishment mechanism
        for representatives. At the moment, this solution seems to be too complicated to be practical.

\end{enumerate}
\section{Lottery}
\label{sec:lottery}
    \paragraph{Lottery contract.}
    Lotteries are handled by a single Ethereum contract which stores information about all the lotteries that are in
    progress. Participants (the payer and payees) send messages to this contract to start the lottery and claim
    the reward. The basic protocol consists of two types of messages: \texttt{lotteryInit}, sent by the payer to
    establish a lottery after the task is completed, and \texttt{lotteryWinner}, sent by any participant
    (presumably the winner) after the winner is determined to transfer the reward to the winner's account and end
    the lottery.

    \paragraph{Starting the lottery.}
    To start a new lottery, the payer creates a lottery description $L$ and calculates its hash $h(L)$. The
    description contains a unique lottery identifier, Ethereum addresses $a_1,\,\ldots,\,a_n$ of each payee and
    payment values $v_1,\,\ldots,\,v_n$ due to each payee, possibly together with some other data. The payer then
    sends an \texttt{lotteryInit} message to the contract to announce the lottery. The message contains $h(L)$ and
    also transfers the task value from the payer to the contract (in Ethereum, each contract is associated with an
    account). The contract stores $h(L)$ in the Ethereum persistent storage. Since $L$ contains a unique lottery
    identifier, $h(L)$ is also a unique value and may be used as a key when storing and retrieving various lottery
    data to/from Ethereum persistent storage.

    The payer also announces the lottery description $L$ to the Golem network, so that every interested node can
    verify that its payment $v_i$ has the correct value and check that $h(L)$ is indeed written to the Ethereum
    storage.

    \paragraph{Determining the winner.}
    The winner of the lottery can be uniquely determined from its description $L$ and some random value $R$ that
    is not known to any party at the moment the lottery is started. We assume that $L$ and the timestamp $t_0$ of
    the moment at which $h(L)$ is written to the Ethereum storage determine another timestamp $t_R$ of some future
    moment at which $R$ will be revealed to everyone. From this moment on, $R$ will also be available to the lottery
    contract.

    A standard way to implement this rather abstract assumption in a blockchain setting is to use the hash of some
    future block as a random value. When the lottery is initialized the contract increments the number of the current
    Ethereum block by a fixed number, determining the number of a deciding block. The hash of this future block will
    be used to determine the lottery winner. There is a problem though with implementing this simple solution in
    Ethereum (see Section \ref{sec:problem256} for details).

    \paragraph{Claiming the reward.}
    Lottery reward may be claimed by sending a \texttt{lotteryCheck} message to the lottery contract, with the
    full lottery description $L$. The contract then calculates $h(L)$ and checks that it exists in the storage,
    which indicates that the lottery has started but the reward has not been claimed yet.
    If $t_R$ has elapsed and $R$ is available, the contract computes the winner's address from $L$ and $R$ and
    transfers the reward from the contract's account to the winner. At this point, the contract removes $h(L)$
    from the storage.

    The problem with the basic protocol is that the size of $L$ is proportional to the number of payees and whoever
    sends the \texttt{lotteryCheck} message has to pay for sending $L$ and processing it by the contract.
    Taking into account that sending a byte of data in a message costs 68 gas units we estimate that sending and
    processing the message will require at least 2000 gas units per payee. This would defeat the whole lottery
    approach, which was proposed in order to avoid paying transaction fees proportional to the number of payees.

    Fortunately, in order to verify that $a_i$ is the winner the contract does not need to examine whole $L$.
    If the list of payees with their payment values is stored in a data structure called the Merkle tree, it is
    enough to send and examine an amount of data proportional to the logarithm of the number of payees to verify
    that $a_i$ is the winner (see Appendix \ref{sec:lottery-description} for details). Therefore instead of sending
    full lottery description with \texttt{lotteryCheck} we can send the address $a_i$ of the winner and just enough
    data to verify it.

\subsection{Optimistic approach}
    If there is a lot of payees, the cost of \texttt{lotteryCheck} may still be high even when only a part of the
    lottery description is required to compute the winner. We propose an extension of the basic protocol that allows
    the parties to avoid sending the \texttt{lotteryCheck} message altogether. Instead, a payee can claim to be the
    winner by sending a new \texttt{lotteryWinner} message containing only $h(L)$. The contract cannot verify the
    claim and only checks that $h(L)$ exists in the storage. Then the contract stores additional data along with
    $h(L)$, namely the address $a_i$ of the claimed winner and a computed deadline timestamp $t_d$ which determines
    when the reward may be paid to $a_i$.

    \paragraph{Claim verification.}
    Until the deadline $t_d$ elapses anyone can reveal the true winner by sending a \texttt{lotteryCheck} message with
    winner address $a_i$ and enough data to validate it. Sending a \texttt{lotteryCheck} message after a
    \texttt{lotteryWinner} message should only happen if the payee that claimed to be the winner is caught cheating,
    which we hope will occur rarely (hence the name "optimistic approach"). To encourage peers to reveal cheaters,
    and to punish dishonest participants, we provide a mechanism of deposits. A payee sending the \texttt{lotteryWinner}
    message must transfer a deposit which is paid back if no one protests before the deadline elapses.
    This deposit is a reward for proving that the payee claiming to be the winner is cheating.

    \paragraph{Paying out the reward.}
    After the deadline $t_d$ elapses, anyone may send a message \texttt{lotteryPayout} with $h(L)$ as the argument.
    If $h(L)$ exists in the storage and the address of the payee claiming to be the winner is set, the reward is
    transferred to this address and the lottery data is erased from the storage.

    Under the assumption that all participants are honest, the cost of a lottery is constant, independent of the number
    of payees.
    An important property of the protocol described above is that after the random value is revealed, the payee who
    determines to be the winner may calculate the cost of a \texttt{lotteryCheck} message and decide whether to perform
    the basic protocol by sending \texttt{lotteryCheck} and obtain the reward immediately (modulo the time required by
    the Ethereum network to confirm the transaction), or to follow the extended protocol by sending the
    \texttt{lotteryWinner} with a deposit, wait until the deadline elapses and send the \texttt{lotteryPayout} message
    to get the lottery reward.
\subsection{The problem of 256 past blocks}
    \label{sec:problem256}
    Recall that we planned to use the hash of a future Ethereum block (deciding block) to determine who wins
    the lottery. More precisely, we assume that when the lottery is started the contract computes the number
    of the deciding block and stores this number together with other lottery data. Later on, when a
    \texttt{lotteryCheck} is sent the contract checks if a block with the required number has already been
    produced and if so, uses its hash to calculate who is the winner. Now, a problem with this approach is that in
    Ethereum, contracts can only access hashes of the last 256 blocks. Since a block is created roughly each 15
    seconds, this gives us only about an hour during which the hash value is directly available from the contract.
    There is no possibility of scheduling a message for sending at a later time in Ethereum, other than relying on
    a third-party contract (for example Alarm Clock Contract \cite{ALARM}), so during this hour some party has
    to explicitly send either a \texttt{lotteryCheck} message that will use the hash value to compute the winner,
    or some other message that will prompt the contract to store the hash value for later use.

    \paragraph{Capturing the hash of the deciding block.}

    To address this issue we extend the protocol with a \texttt{lotteryCaptureHash} message the result of which is to
    store the hash of the deciding block. To encourage sending this message soon after the deciding block is generated
    we require that the sender of a \texttt{lotteryInit} message deposits an additional amount on the contract's
    account, together with the lottery reward. The deposit is returned to the payer if the payer sends a
    \texttt{lotteryCaptureHash} message within 128 blocks after the deciding block is generated.
    Between 128 and 256 blocks after the deciding block anyone may send this message and claim the payer's deposit.

    As a last resort, if the hash of the deciding block is not stored and at least 256 more blocks are generated,
    a \texttt{lotteryCaptureHash} message will store the hash of the unique available block with the number congruent
    modulo 256 to the number of the original deciding block. In this case, the payer's deposit will remain in the
    contract's account. Thus instead of trying to retrieve the lost hash we allow the winner to be changed.

    If we could overcome the problem of 256 blocks then a payer deposit and \texttt{lotteryCaptureHash} message would
    be pointless and protocol would become simpler. An alternative solution would be to use a separate contract
    for providing hashes of past blocks. See Section \ref{sec:randomness} for more remarks on alternative sources
    of randomness.
\subsection{Lottery agents}
    We have to account for a situation when the lottery winner goes offline for a longer period of time and is
    unable to claim the reward or simply does not want or cannot pay for sending a \texttt{lotteryCheck} or
    \texttt{lotteryWinner} message. In such cases a third party may serve as \textbf{lottery agent} sending messages
    instead of the winner while keeping part of the reward. In particular, after a fixed amount of time elapses
    since the time when the random value $R$ is revealed (the time when the deciding block is produced) and nobody has
    claimed to be a winner, anyone may send the \texttt{lotteryCheck} message which will transfer part of the reward,
    say 10\%, to the sender of the message and the rest to the winner, and will erase the contract from Ethereum
    storage.
\subsection{Lottery Protocol Specification}
    The lottery contract is implemented in Solidity \cite{SOLIDITY}, a high-level language compiled to machine code of
    the Ethereum Virtual Machine\cite{ETHERDEV}. A contract in Solidity consists of a number of functions, each
    function is a entry point to the contract code. Below we list Solidity functions that represent messages in the
    lottery protocol. The types of message arguments are as follows:
    \begin{itemize}
        \item \texttt{uint} (an alias for \texttt{uint256}) is the type of 256 bit unsigned integers,
        \item \texttt{bytes32} is the type of fixed-size arrays of 32 bytes,
        \item \texttt{address} is the type of 160 bit Ethereum addresses.
    \end{itemize}
    In all the functions, the first parameter (\texttt{bytes32 lotteryHash}) is used to specify the lottery on which
    the function operates. Money is denominated in wei (1 wei = 1.0e-18 ether) and represented as \texttt{uint} values.

    \paragraph{Contract functions.}

    \begin{itemize}
        \item \texttt{function lotteryInit(bytes32 lotteryHash)}

            The sender Initializes a lottery. Lottery value is transferred to the lottery contract and a structure
            with lottery data is created in the storage. \texttt{lotteryHash} is the value used as a key to retrieve
            lottery data from the contract storage (see Appendix \ref{sec:lottery-description} for information how it
            is computed).

            Estimated cost: 85 000 gas.

        \item \texttt{function lotteryWinner(bytes32 lotteryHash)}

            The sender claims to be the winner. Estimated cost: 85 000 gas.

        \item \texttt{function lotteryCaptureHash(bytes32 lotteryHash)}

            The message saves the hash of maturity block if it is possible. The payer should send this message
            within 128 blocks since maturity in order to get back the payer deposit.
            Later but within 256 blocks anyone else can send this message and get the payer deposit as a reward.

            Estimated cost: 30 000 gas.

        \item \texttt{function lotteryCheck(bytes32 lotteryHash, uint256 uid, address winner, }\\
            \hphantom{} \hskip 5em  \texttt{uint32 rangeStart, uint32 rangeLength, bool[] path, bytes32[] values)}

            The sender specifies an address and the data required to verify that it is the winner's address.
            If the address is verified, the reward is transferred and the lottery data is erased from the storage.

            Estimated cost: $40\;000 + 2\;700 \cdot \log_2(N)$ gas.

        \item \texttt{function lotteryPayout(bytes32 lotteryHash)}

            The sender wants to receive his claimed reward after deadline elapsed. Transfers the reward and erases
            lottery data from the storage.

            Estimated cost: 40 000 gas.

    \end{itemize}

    \paragraph{LotteryData struct}

    For every lottery that is in progress the following Solidity structure is stored in the Ethereum storage:

    \begin{tabular}{l}
        \texttt{struct LotteryData \{}\\
        \qquad\texttt{uint value;}\\
        \qquad\texttt{uint maturity;}\\
        \qquad\texttt{uint deadline;}\\
        \qquad\texttt{uint randVal;}\\
        \qquad\texttt{address payer;}\\
        \qquad\texttt{address winner;}\\
        \texttt{\}}
    \end{tabular}

    The meaning of the fields is as follows:
    \begin{itemize}
        \item \texttt{value} is the lottery reward (in wei),
        \item \texttt{maturity} is the number of the deciding block ($t_R$),
        \item \texttt{deadline} is the timestamp at which the user that claimed to be the winner may collect
            the reward ($t_d$),
        \item \texttt{randVal} is the random value determined from the hash of the deciding block ($R$),
        \item \texttt{payer} is the address of the user that created the lottery,
        \item \texttt{winner} is the address of the user that claimed to be the winner.
    \end{itemize}

    The contract will also have an \texttt{uint} variable \texttt{golem\_dep} (Golem's deposit) which will store
    the total amount collected in the contract from commissions and lost deposits. This amount will be eventually
    redeemed by equity token holders (all Golem application owners).

    The syntax of Solidity is similar to JavaScript. In particular, if lotteries is a mapping of uints to structs of type LotteryData then the statement
    \begin{center}
        \texttt{lotteries[lotteryHash].deadline = t}
    \end{center}

    will update the field \texttt{deadline} in the \texttt{LotteryData} struct for a lottery with hash
    \texttt{lotteryHash}. One difference from JavaScript is that in Solidity a \texttt{mapping} is a total
    function, in particular \texttt{lotteries} initially maps all possible key values to
    a \texttt{LotteryData} struct with all fields initialized to 0. To check if a lottery data has been initialized
    we will check if \texttt{lotteries[lotteryHash].value} is nonzero.


\paragraph{Protocol}

    In the description of messages below the semantics of the \textbf{Preconditions} and \textbf{Effects} clauses is
    that if all the conditions specified in \textbf{Preconditions} are satisfied then the contract's state is changed
    as described in \textbf{Effects}. Otherwise, the message has no effect on the state of the contract.
    \begin{enumerate}
        \item The payer negotiates payment with every payee independently. Let $v_i$ be the amount that $i$-th payee
            expects after this step and let $a_i$ be the address of the $i$-th payee.
        \item The payer calculates the transaction value $v = \sum_i v_i$, calculates the probability of winning
            for each payee and composes a lottery description $L$ containing these data (see Section
            \ref{sec:lottery-description} for details). The payer then sends $L$ to every payee. The payees check that $L$
            is correct. In particular, they may verify that the expected lottery reward for the $i$-th payee is equal
            to $v_i$.
        \item The payer sends a \texttt{lotteryInit} message to the contract with the lottery hash $h(L)$ as the
            argument. The lottery hash can be calculated by anyone who knows $L$ and is used as a unique lottery
            identifier throughout the protocol (see Section \ref{sec:lottery-description} for information how $h(L)$ is
            computed).\\
            \textbf{Preconditions}: no \texttt{LotteryData} struct is associated with $h(L)$ in the contract's
            storage. This is equivalent to the condition
            \begin{center}
	            \texttt{lotteries[lotteryHash].value == 0}
	        \end{center}
            \textbf{Effects}: The message transfers the lottery value $v$ and the payer deposit (additional 10\%
            of $v$) to the contract and stores the \texttt{LotteryData} struct for the new lottery. The field value
            is set to $10/11$ of the value transferred by the \texttt{message} and and the remaining 1/11 of this value
            is treated. Field \texttt{maturity} is initialized with argument $m$ and \texttt{payer} is initialized with
            the sender's address. The remaining fields are initialized with zeros.
            \begin{center}
            \texttt{lotteries[lotteryHash] = LotteryData(value, block.number + maturity, 0, 0, msg.sender, 0)}
            \end{center}
            At this point any payee can calculate the hash of the lottery on its own and can verify if it exists in
            the contract's storage. It can be done by querying the (locally available) Ethereum state.
        \item Anyone (not only a payer) can send a \texttt{lotteryCaptureHash} message with argument $h(L)$ to write
            the hash of the deciding block as the lottery seed.\\
            \textbf{Preconditions}:
            \begin{enumerate}
                \item The value field of the \texttt{LotteryData} for the lottery has been set (meaning that the
                lottery has been initialized but not finalized yet).
                    \begin{center}
	                    \texttt{lotteries[lotteryHash].value != 0}
                    \end{center}
                \item The lottery \texttt{randVal} has not been set yet.
                    \begin{center}
	                    \texttt{lotteries[lotteryHash].randVal == 0}
	                \end{center}
                \item The maturity has elapsed (the stored maturity number is less than current pending block number).
                    \begin{center}
                	    \texttt{lotteries[lotteryHash].maturity < block.number}
                	\end{center}
	        \end{enumerate}
        \textbf{Effects}:
        \begin{enumerate}
            \item  If the difference of the current block number and maturity is not greater than 128 then the hash of
                the deciding block is written as the \texttt{randVal} and the payer deposit is transferred to the payer.
                \begin{center}
                    \begin{tabular}{l}
                        \texttt{lottery.randVal = random(lottery.maturity);}\\
                        \texttt{lottery.payer.send(countPayerDeposit(lottery.value));}
                    \end{tabular}
                \end{center}
            \item If the difference of the current block number and maturity is greater than 128 but not greater than
                256 then  the hash of the deciding block is written as the \texttt{randVal} the payer deposit is
                transferred to the message sender.
                \begin{center}
                    \begin{tabular}{l}
                        \texttt{lottery.randVal = random(lottery.maturity);}\\
                        \texttt{msg.sender.send(countPayerDeposit(lottery.value));}
                    \end{tabular}
                \end{center}
            \item If the difference of the current block and maturity is greater than 256 then the hash of the block
            with number maturity$+k \cdot 256$, where $k$ is the greatest possible integer, is written to the contract
            as the \texttt{randVal} and the payer deposit remains in the contract (a contract owner gets it).
            \begin{center}
                \begin{tabular}{l}
                    \texttt{lottery.randVal = random(changeMaturity(lottery.maturity));}\\
		            \texttt{golem\_dep += countPayerDeposit(lottery.value);}
		        \end{tabular}
		    \end{center}
        \end{enumerate}
        \item Anyone (not only a payee) can send a \texttt{lotteryWinner} message to claim that he is the winner.
            The message contains the lottery hash and requires to transfer a winner deposit.\\
            \textbf{Preconditions}:
            \begin{enumerate}
                \item The maturity stored for the lottery has been set.
                    \begin{center}
                        \texttt{lotteries[lotteryHash].maturity != 0}
                    \end{center}
                \item The maturity has elapsed.
                    \begin{center}
		                \texttt{lotteries[lotteryHash].maturity < block.number}
		            \end{center}
                \item Nobody has claimed to be the winner yet.
                    \begin{center}
                        \texttt{lotteries[lotteryHash].winner == 0}
                    \end{center}
            \end{enumerate}
            \textbf{Effects}:
            \begin{enumerate}
                \item The contract sets the claimed winner to the sender's address and sets deadline.
                    \begin{center}
                        \begin{tabular}{l}
                		    lottery.winner = msg.sender;
                            lottery.deadline = now + deadline;
                        \end{tabular}
                    \end{center}
                \item If the \texttt{randVal} is not set than calls \texttt{lotteryCaptureHash} message.
                    \begin{center}
                        \texttt{lotteryCaptureHash(lotteryHash)}
                    \end{center}
            \end{enumerate}
        \item Anyone (not only a payee or the payer) can send a \texttt{lotteryCheck} message. The message contains
            the lottery hash and the data required to compute the winner (see Section \ref{sec:lottery-verification}
            for details).\\
            \textbf{Preconditions:}
            \begin{enumerate}
                \item The maturity stored for the lottery has been set
                    \texttt{lotteries[lotteryHash].maturity != 0}
                \item The maturity has elapsed.
	                \texttt{lotteries[lotteryHash].maturity < block.number}
	        \end{enumerate}
	        \textbf{Effects:}
	        \begin{enumerate}
                \item If the \texttt{randVal} is not set than call \texttt{lotteryCaptureHash} message.
                    \begin{center}
                        \texttt{lotteryCaptureHash(lotteryHash)}
                    \end{center}
                \item If 28800 blocks has been added to the blockchain after the deciding block (this would take
                    approx. 5 days) then 10\% of the transaction value $v$ is transferred to the sender (which is treated
                    as a lottery agent) and the transaction value is reduced to 90\% of its previous value.
                    \begin{center}
                        \texttt{msg.sender.send(countAgentProvision(lottery.value));}
                    \end{center}
                \item If the winner field is set to 0 in the lottery data than the transaction value is sent to the
                    winner.
                    \begin{center}
                        \texttt{winner.send(lottery.value);}
                    \end{center}
                \item If the winner field is equal to the winner address then the transaction value and the winner
                    deposit are transferred to the winner.
                    \begin{center}
			            \texttt{winner.send(lottery.value + countWinnerDeposit(lottery.value));}
			        \end{center}
			    \item If the winner field is not equal to the winner address then the transaction value is
			        transferred to the winner, winner deposit is transferred to the sender.
			        \begin{center}
			            \texttt{msg.sender.send(countWinnerDespoit(lottery.value));}
			        \end{center}
                \item The lottery is erased from the contract.
                    \begin{center}
                        \texttt{delete lotteries[lotteryHash]}
                    \end{center}
            \end{enumerate}
        \item Anyone can send a \texttt{lotteryPayout} message to the contract. The message contains the lottery hash.\\
            \textbf{Preconditions:}
            \begin{enumerate}
                \item The deadline has elapsed.
                    \begin{center}
		                \texttt{lotteries[lotteryHash].deadline < now}
		            \end{center}
                \item Somebody has claimed to be a winner (winner isn't set to zero).
                    \begin{center}
                        \texttt{lotteries[lotteryHash].winner != 0}
                    \end{center}
            \end{enumerate}
            \textbf{Effects:}
            \begin{enumerate}
                \item The transaction value and the deposit are transferred to the claimed winner.
                    \begin{center}
			            \texttt{lottery.winner.send(lottery.value + countWinnerDeposit(lottery.value))}
			        \end{center}
                \item Transaction is completed and the lottery data is erased from the contract.
                    \begin{center}
                        \texttt{delete lotteries[lotteryHash]}
                    \end{center}
           \end{enumerate}
    \end{enumerate}
\subsection{Cost comparison}
\subsection{Source of randomness for the Lottery implementation}
\label{sec:randomness}
    In our lottery implementation we use the hash of a future block as the source of randomness. This solution has
    two drawbacks. First, there is a risk that the random value may be manipulated by Ethereum miners. That is,
    a miner may decide whether to publish a newly found block after looking at the block hash and calculating
    the winning address. Clearly, if the lottery reward is small compared to the reward for finding a new block
    then it is not profitable to withhold a block to change the odds of winning a lottery. Thus one way to circumvent
    this problem is to limit the maximum lottery reward and in case the task value exceeds this limit, split payees
    into several groups and organize a separate lottery for each group, with a fraction of the task value as the
    reward.

    The second problem is due to the fact that only the hashes of past 256 blocks are available in the contract
    so we have to complicate the protocol and introduce the mechanism of deposits to encourage the participants
    to capture the hash of the deciding block.

    Let us consider some alternatives to using block hashes as the source of randomness.

    \paragraph{Not using any external source of randomness at all.}
    Let us consider a protocol in which the winner is determined using only data provided by a group of participants,
    in a peer-to-peer environment, without using any external sources (including hashes of future block) and without
    relying on any trusted third party. That is, each participant reveals a piece of data and once all data is public
    the winner is determined in a provable way.

    The protocol needs to deal with situations in which some participants fail to reveal their data. This may be due
    to communication failure or may be done on purpose. Assume one participant delayed revealing her data until all
    other data are revealed. This participant has now full information required to determine who the winner is,
    and may decide to quit the protocol if she decides that its outcome is not to her advantage\footnote{We assume
    the participants commit to their data before the protocol starts, for example using hash functions,
    so changing the data at this point is not an option.}.

    In this situation the procedure may be either (1) to determine the winner using only the available data or (2)
    to repeat the whole protocol without the participant who quit. The first case means that input from all
    participants is not required for completing the protocol which may make the whole procedure vulnerable to
    various denial-of-service attacks that prevent participants from providing their data.

    In the second case, participants may collude to affect the outcome. Suppose there is a group of participants
    cooperating to increase the chance of one of them winning. Now, one of the group may delay revealing the last
    piece of data and quit if the winner would not be a member of the group, and the procedure would need to be
    repeated. Obviously, such behavior increases the chances that a group member will finally turn out to be the
    winner and thus makes the whole protocol unfair\footnote{We may consider a modification of the protocol in which
    the order in which the participants reveal their data is fixed in some fair way. However, there will still be
    a possibility that the last participant to reveal his data is a member of the group.}.

    Note that participants may be discouraged from quitting the lottery for example by losing deposits they have to
    make earlier. This would make the whole lottery protocol more complicated and would require some further design
    decisions, for example how big should the deposit be, what if a participant does not have money to pay the deposit
    and so on.

    \paragraph{Using a separate contract as source of randomness.}
    Another possibility is to use a separate contract to provide random values. We already mentioned RANDAO
    (\cite{RANDAO}, \cite{RANDAO2}), other similar third party solutions may appear in future.
    We may also consider implementing our own custom contract for storing hashes of past blocks.
    The idea is to consider only hashes of blocks with numbers divisible by 100 (for instance) and remember last 100
    (for instance) such hashes. This means that a new random value can be generated roughly each 25 min
    (time to generate 100 blocks) and the contract may reproduce values from the past 100 days
    (time to generate 10 000 blocks). The contract stores an array of 100 hashes and has two functions:
    \texttt{feed()} for writing the hash of the most recent block with number divisible by 100, and
    \texttt{get(blocknum)} which returns the hash for the latest block with number divisible by 100 less or
    equal to \texttt{blocknum}.

    \label{sec:hundred}

    \begin{tabular}{l}
        \texttt{contract OneHundredHashes \{}\\
        \\
	    \qquad \texttt{bytes32[100] hashes;}\\
        \\
	    \qquad\texttt{function feed() external \{}\\
    	\qquad\qquad \texttt{uint8 index = uint8((block.number / 100) \% 100);}\\
    	\qquad\qquad \texttt{hashes[index] = block.blockhash(block.number - block.number \% 100);}\\
	    \qquad\texttt{\}}\\
        \\
	    \qquad \texttt{function get(uint blocknum) external returns (bytes32) \{}\\
        \qquad\qquad \texttt{uint8 index = uint8((blocknum / 100) \% 100);}\\
    	\qquad\qquad \texttt{return hashes[index];}\\
	    \qquad\texttt{\}}\\
    \texttt{\}}
    \end{tabular}

    The cost of a call to feed is 26 000 gas (21 000 for a transaction + 5 000 for updating storage) and feed should
    be called at least every 100 blocks, that is, roughly 20 000 times a year. With the current price of gas
    and Ether, the cost of running the contract is approx \$20 per year. The cost of contract initialization and
    storing first 100 hashes is negligible (well below \$1). The cost of calling get is slightly above 21 000 gas
    (reading from storage is cheap compared to writing).

    Obviously the contract relies on an external service that calls feed once in every 25 min.

\section{Conclusion}
\begin{thebibliography}{9}
\bibitem{ALARM} Ethereum Alarm Clock contract website, \url{http://www.ethereum-alarm-clock.com/}.
\bibitem{ANDRESEN} Andresen, Gavin. \textit{"Back-of-the-envelope calculations for marginal cost of transactions"},
    \url{https://gist.github.com/gavinandresen/5044482}.
\bibitem{BITCOIN} Nakamoto, Satoshi. \textit{"Bitcoin A Peer-to-Peer Electronic Cash System"}, 2008,
    \url{https://bitcoin.org/bitcoin.pdf}.
\bibitem{BITCOINJ} \textit{"Working with micropayment channels"},
    \url{https://bitcoinj.github.io/working-with-micropayments}.
\bibitem{BITFEE} \textit{"Bitcoin Transaction Fees Explained"} web page, last update 2014, \url{http://bitcoinfees.com/}.
\bibitem{BUTERIN} Buterin, Vitalik. "Scalability, Part 1: Building on Top", Ethereum Blog, September 17, 2014,
    \url{https://blog.ethereum.org/2014/09/17/scalability-part-1-building-top/}.
\bibitem{CHANGETIP} ChangeTip website, \url{https://www.changetip.com/}.
\bibitem{COINTIP} \textit{"Shutting Down Coinbase Tip Button"}, 2015,
    \url{https://blog.coinbase.com/2015/02/10/shutting-down-the-coinbase-tip-button/}.
\bibitem{ETHERDEV} \textit{Ethereum Development Tutorial}, last update May 24, 2015,
    \url{https://github.com/ethereum/wiki/wiki/Ethereum-Development-Tutorial}.
\bibitem{ETHEREUM} \textit{"Ethereum White Paper"}, \url{https://github.com/ethereum/wiki/wiki/White-Paper}.
\bibitem{FRS} Board of Governors of the Federal Reserve System, Press Release, June 29, 2011,
    \url{http://www.federalreserve.gov/newsevents/press/bcreg/20110629a.htm}.
\bibitem{JAIN} Jain, Mohit, Siddhartha Lal and Anish Mathuria. \textit{"A Survey of Peer-to-peer Micropayment Schemes},
    2008, \url{http://www.dgp.toronto.edu/~mjain/P2P_Micropayment-2008.pdf}.
\bibitem{KASKALOGLU}Kaşkaloğlu, Kerem, \textit{"Near Zero Bitcoin Transaction Fees Cannot Last Forever"},
    The International Conference on Digital Security and Forensic(DigitalSec2014), 91-99, June, 2014,
    \url{http://sdiwc.net/digital-library/near-zero-bitcoin-transaction-fees-cannot-last-forever.html}.
\bibitem{RANDAO} McKinnon, Dennis. RANDAO repository,
    \url{https://github.com/dennismckinnon/Ethereum-Contracts/tree/master/RANDAO}.
\bibitem{RANDAO2} Qian, Yaucai. RANDAO repository, \url{http://github.com/randao}.
\bibitem{RIVEST} Rivest, Ronald. \textit{"Electronic Lottery Tickets as Micropayments"},
    "In Financial Cryptography", 307--314, Springer Verlag, 1997,
    \url{https://people.csail.mit.edu/rivest/pubs/Riv97b.pdf}.
\bibitem{SOLIDITY} \textit{Solidity Documentation}, \url{https://ethereum.github.io/solidity/docs/home/}.
\bibitem{SHA3} FEDERAL INFORMATION PROCESSING STANDARDS PUBLICATION,
    \textit{"SHA-3 Standard: Permutation-Based Hash and Extendable Output Functions"}, FIPS PUB 202,
    \url{http://nvlpubs.nist.gov/nistpubs/FIPS/NIST.FIPS.202.pdf}.
\bibitem{VISHNUMURTHY} Vishnumurthy, Viviek, Sangeeth Chandrakumar and Emin G\:{u}n Sirer, \textit{"KARMA: A Secure
    Economic Framework for Peer-to-Peer Resource Sharing"}, 2003,
   \url{http://cs.brown.edu/courses/csci2950-g/papers/karma.pdf}.
\bibitem{WHEELER} Wheeler, David. \textit{"Transactions Using Bets"},
    "In proceedings Fourth Cambridge Workshop on Security Protocols", 82--92, Springer, 1996.
\bibitem{WOOD} Wood, Gavin. "Ethereum: A Secure Decentralized Generalised Transaction Ledger", 2014,
    \url{http://gavwood.com/paper.pdf}.
\end{thebibliography}
\appendix
\section{Dictionary}
\section{Contract code}
\section{Lottery description}
\label{sec:lottery-description}
    Here we show how to store lottery data in a Merkle tree, so that the contract can verify that a specified payee
    is indeed the winner in the number of steps proportional to the logarithm of the number of payees.
    We assume a fixed cryptographic hash function [ref] that will be used for the payment protocol.
    For a binary string $B$, $hash(B)$ will denote the result of the hash function applied to $B$.
    For a sequence $B_1, ..., B_n$ of binary strings, $hash(B_1, ..., B_n)$ will denote the the result of
    the hash function applied to the concatenation of $B_1, ..., B_n$. In the concrete Ethereum implementation SHA-3
    with 256-bit output will be used.

    In the following we assume that the task computation may be divided to at most $2^S$ subcomputations or,
    in other words, that the payment to each payee is a multiple of $V/2^S$, where $V$ is the value of the whole task.
    A large value of $S$ will allow us to split the task in a fine grained subtasks but will make the lottery
    description bigger.

    Let $N$ be the number of payees and, for each $i$ in $\{1, .. N\}$, let $r_i$ be such that the value due to
    the $i$-th payee is

    \begin{displaymath}
        v_i = r_i \cdot \frac{V}{2^S},
    \end{displaymath}

    that is

    \begin{displaymath}
        r_i = 2^S \cdot \frac{v_i}{V}.
    \end{displaymath}

    Since $\sum_i v_i = V$, we also have $\sum_i r_i = 2^S$.

    For each $i$ in $\{1, ..., N\}$, let $R_i$ denote the $\sum_{j<i} r_i$.

    Given a number $x$ in $\{0, 1 ,... ,2^S-1\}$, the winner of the lottery is the unique index $i$ in $\{1, ..., N\}$
    such that
    \begin{displaymath}
        R_i \leq x < R_i+r_i.
    \end{displaymath}

    \begin{exmp}[with S=3, N=5]
        Let $r_1 = 2$, $r_2 = 3$, $r_3 = 1$ and $r_4 = 2$. Then we have $R_1 = 0$, $R_2 = 2$, $R_3 = 5$ and $R_4 = 6$.\\
        Now,
        \begin{displaymath}
            \begin{array}{c}
                1 \text{ wins if} \quad 0 \leq x < 2\\
                2 \text{ wins if} \quad 2 \leq x < 5\\
                3 \text{ wins if} \quad 5 \leq x < 6\\
                4 \text{ wins if} \quad 6 \leq x < 8
            \end{array}
        \end{displaymath}
    \end{exmp}

    \begin{dfnt}[lottery description]
        A \textbf{lottery description} $L$ is a data structure that contains all relevant data for a lottery instance,
        such as the address of the Golem node that announces the lottery, a timestamp (which together uniquely identify
        a lottery), task value $V$, the list of payee addresses $a_1,\,\ldots,\, a_N$ and the corresponding list
        $p_1, \,\ldots,\, p_N $of probabilities of winning the lottery for each of the payees. $L$ may also contain
        some other data required by the implementation, the exact details are not relevant. We assume that all parties
        agree on the format used for lottery descriptions and are able to verify that L is valid. By $B(L)$ we denote
        the binary representation of $L$ (i.e. encoding of $L$ as a sequence of bits).
    \end{dfnt}


    \begin{dfnt}[payment list]
        A \textbf{payment list} for $L$ is a sequence $P(L) = ((a_1, R_1, r_1), \,\ldots, (a_N, R_N, r_N))$ where the values
        $R_1,\,\ldots,\, R_N$ and $r_1,\,\ldots,\, r_N$ are computed from task value $V$ and probabilities
        $p_1,\,\ldots,\,p_N$ as described above. To restrict our attention we assume that $R_i$ and $r_i$ are 32 bit
        words (thus $S = 32$). The definitions below can be adjusted to other values of $S$ in a straightforward way.
    \end{dfnt}

\subsection{Lottery verification}
    \label{sec:lottery-verification}
    Now, in order to verify that $a_i$ is the winner of a lottery described by $L$ for a given random value $R$,
    one has to make sure that $P(L)$ contains a tuple $(a_i, R_i, r_i)$ such that $R_i \leq R < R_i + r_i$ holds.
    This can be done by proposing the tuple $(a_i, R_i, r_i)$ and:
    \begin{enumerate}
        \item checking that it satisfies the required inequalities,
        \item proving that it is an element of $P(L)$.
    \end{enumerate}

    (1) is trivial and (2) can be done by iterating over $P(L)$ until $(a_i, R_i, r_i)$ is found. However,
    doing so in the lottery contract would require sending whole $P(L)$ in a message (or half of $P(L)$ on average,
    if the triples in $P(L)$ are sorted) which would incur a substantial cost for the sender. Fortunately,
    the verification may be performed in the number of steps and with the size of data proportional to the logarithm
    of $N$ by encoding information in $P(L)$ in a \textbf{Merkle tree}, that is a full binary tree in which every inner node
    is labeled by the hash of the labels of its children.

    In the following we make use of the standard identification of binary trees with nonempty prefix-closed sets
    of sequences over $\{0,1\}$. That is, $T \subseteq \{0,1\}^*$ is a binary tree if $T \neq \empty$ and for every
    $n \in \{0,1\}^*$ and $b \in \{0,1\}$, if $nb \in T$ then $n \in T$. Here, the empty sequence is the
    root of $T$ and $n0$ and $n1$ are the left and the right child of $s$,respectively.
    $T$ is \textbf{full} if every inner node has two children, that is if for every $n \in T$ we have $n0 \in T \iff n1 \in T$.
    A \textbf{labeled binary tree} is a binary tree $T$ and a function $l$ from $T$ to some fixed set of labels.
    Finally, a Merkle tree is a \textbf{labeled full binary tree} $(T, l)$ with the la belling function
    $l:\; T \rightarrow \{0,1\}^{256}$ satisfying:
    \begin{displaymath}
        l(n) = hash(l(n0), l(n1))
    \end{displaymath}
    for every $n \in \{0,1\}^*$ such that $n0 \in T$.

    A pleasant property of Merkle trees is that to prove that a given tree contains a node with specific label $w$
    we only need to examine the amount of data proportional to $\log(d)$, where $d$ is the depth of the specified node.

    Let us fix a Merkle tree $T$ and a node $n = b_1,\,\ldots,\,b_d$ with label $w = l(n)$ and let $w_1,\,\ldots,\,w_d$
    be a sequence of 256 bit words such that for each $i \in \{1,\,\ldots,\,d\}$, if $b_i = 0$ then $w_i$ is the label
    of the right child of the node $b_1,\,\ldots,\, b_{i-1}$ and otherwise $w_i$ is the label of its left child
    (that is, $w_i = l(b_1, \,\ldots,\,b_{i-1}, 1 - b_i)$).

    Let $h_0,\,\ldots,\, h_d$ denote the labels of the nodes on the path from the root to $n$. That is, $h_0$ is
    the label of the root, $h_d = w$ and in general, $h_i = l(b_1,\,\ldots,\,b_i)$. From the property of Merkle trees
    we have, for every $i \in \{1,\,\ldots,\,d\}$:
    \begin{itemize}
        \item if $b_i = 0$ then $h_{i-1} = hash(h_i, w_i)$
        \item if $b_i = 1$ then $h_{i-1} = hash(w_i, h_i)$
    \end{itemize}
    This means that given $n,\, w$ and $w_1,\,\ldots,\,w_d$ we may compute the values of $h_i$, starting from $h_d = w$
    and climbing the tree up to $h_0$ which is the label of the root. Since hash is assumed to be collision resistant
    (\cite{SHA3}), finding $n$, $w$ and $w_1,\,\ldots,\,w_d$ for which the above procedure yields the given
    hash value $h_0$ is considered extremely difficult. Therefore, for all practical purposes, proposing the values
    $n,\,w,\,w_1,\,\ldots,\,w_d$, computing the hash $h_0$ and checking that it is equal to the label in the root of
    a Merkle tree $T$ is considered a valid proof of $n$ being a node in $T$ labeled with $w$.

    \begin{figure}
        % Merkle tree example for the micropayments Golem paper
% Requires \usepackage{pgfplots} 

\begin{tikzpicture}[level/.style={sibling distance=8cm/#1}]
\renewcommand{\hash}[1]{\mathrm{hash}({#1})}
\node [rectangle, draw] (c)  {$h_0 = \hash{h_1, w_1}$}
	child { node [rectangle, draw] (l)  {$h_1 = \hash{w_2, h_2}$}  
		child { node [circle, draw, minimum size=0.8cm] (ll) {$w_2$} 
			child { node [circle, draw, minimum size=0.8cm] (lll) {}}
			child { node [circle, draw, minimum size=0.8cm] (llr) {}}
		}
		child { node [rectangle, draw] (B)  {$h_2 = \hash{h_3, w_3}$} 
			child { node [rectangle, draw] (lrl) {$h_3 = w$} 
				edge from parent node[scale=0.83, left=0.1cm] {$b_3= 0$}	
			}
			child { node [circle, draw, minimum size=0.8cm](lrr) {$w_3$}}
			edge from parent node[scale=0.83, right=0.2cm] {$b_2= 1$}	
		}
		edge from parent node[scale=0.83, left=0.4cm, above=0.0cm] {$b_1 = 0$}	
	}
	child { node [circle, draw] (r) {$w_1$}
		child { node [circle, draw, minimum size=0.8cm] (rl) {$$}
		 }
		child { node [circle, draw, minimum size=0.8cm] (rr)  {$$}
		} 
	}
;
\end{tikzpicture}

        \caption{An example Merkle tree.}
        \label{fig:merkle}
    \end{figure}

    The above definitions are illustrated in Figure~\ref{fig:merkle} which shows a Merkle tree with a distinguished
    node $n = 010$. The label $w$ of $n$, together with labels $w_3$, $w_2$ and $w_1$, are used to compute hashes
    $h_3, \,\ldots, \,h_0$.


    Now, going back the lottery setting, a Merkle tree for a lottery description L is a Merkle tree with minimal height
    such that for every $(a_i, R_i, r_i)$ in $P(L)$ there exists a leaf labeled with $hash(a_i, R_i, r_i)$.
    This definition allows many different Merkle trees for a given $L$, so we assume that all lottery participants
    agree on a common algorithm that builds a "canonical" tree $M(L)$ for each $L$, so that every payee may construct
    $M(L)$ after receiving $L$. Alternatively, the sender may send a tree for $L$ together with $L$ to each payee.
    Note that the height of any Merkle tree for $L$, that is the maximum depth of any leaf, is equal to
    $\text{ceil}(\log_2 P)$.

    Now, a \textbf{hash of $L$} is defined as $h(L) = hash(hash(B(L)), h(M(L)))$, where $B(L)$ is the binary encoding of $L$
    and $h(M(L))$ is the label of the root of $M(L)$.

    A winner certificate $C = (a, R, r, b_1, \ldots, b_d, w_1, \ldots, w_d)$ consists of
    \begin{itemize}
        \item 160-bit word $a$ (address of the proposed winner),
        \item 32-bit words $R$ and $r$,
        \item a sequence $b_1, \ldots, b_d$ of bits, with $d \geq 0$,
        \item a sequence $w_1, \ldots, w_d$ of 256-bit words.
    \end{itemize}

    Given $C$ we define $h_d = hash(a, R, r)$ and, for $i = 1, .., d$, $h_{i-1} = hash(h_i, w_i)$ if $b_i = 0$ and
    $h_{i-1} = hash(w_i, h_i)$ otherwise.
    Let $x$ be a 32-bit random value. $C$ is \textbf{valid} for $s$ if
    \begin{displaymath}
        R \leq x < R + r  \text{ and } hash(hash(B(L)), h_0) = h(L).
    \end{displaymath}

    In our concrete implementation a winner certificate is represented by the following Solidity struct:

    \begin{center}
        \begin{tabular}{ll}
            \texttt{struct WinnerCertificate \{} &  \\
		    \qquad \texttt{address winner;} &  \texttt{// winner's address}\\
		    \qquad \texttt{uint32 rangeStart;} & \texttt{// beginning of the range}\\
		    \qquad \texttt{uint32 rangeLength;} & \texttt{// length of the range}\\
		    \qquad \texttt{bool[] path;} & \texttt{// encoding of the leaf as a sequences $b_1, \ldots, b_d$}\\
		    \qquad \texttt{bytes32[] values;} & \texttt{// values $w_1, \ldots, w_d$}\\
    		\texttt{\}}
        \end{tabular}
    \end{center}

    The algorithm that checks the validity of a certificate may be coded as the following Solidity function,
    in which \texttt{rand} is the random value ($x$ in the description above), \texttt{lotteryHash} is $h(L)$
    and \texttt{descriptionHash} is $hash(B(L))$:

    \begin{center}
        \begin{tabular}{l}
            \texttt{function checkCertificate(uint32 rand, bytes32 lotteryHash, bytes32 descriptionHash, }\\
            \qquad\qquad\qquad\qquad\qquad\qquad\qquad \texttt{WinnerCertificate cert) internal returns (bool) \{ }\\
            \qquad \texttt{// Check if random val falls into the range}\\
            \qquad \texttt{if (rand < cert.rangeStart || rand >= cert.rangeStart + cert.rangeLength)}\\
            \qquad\qquad \texttt{return false;}\\
            \\
            \qquad\texttt{// Initially, h is the value stored in the leaf ($h_d$)}\\
            \qquad \texttt{bytes32 h = bytes32(uint256(cert.winner) << 64}\\
            \qquad\qquad\qquad\qquad \texttt{+ uint256(cert.rangeStart) << 32}\\
            \qquad\qquad\qquad\qquad \texttt{+ cert.rangeLength}\\
            \qquad\qquad\qquad\qquad \texttt{);}\\
            \\
            \qquad \texttt{// Update h with hashes $h_{d-1}, \ldots, h_0$}\\
            \qquad\texttt{for (uint i = cert.path.length; i > 0; i--) \{}\\
            \qquad\qquad \texttt{if (cert.path[i-1] == false)}\\
            \qquad\qquad\qquad \texttt{h = sha3(h, cert.values[i-1]);}\\
            \qquad\qquad \texttt{else}\\
            \qquad\qquad\qquad\texttt{h = sha3(cert.values[i-1], h);}\\
            \qquad\texttt{\}}\\
            \\
            \qquad\texttt{// Mix with description hash}\\
            \qquad\texttt{h = sha3(descriptionHash, h);}\\
            \\
            \qquad\texttt{return h == lotteryHash;}\\
            \texttt{\}}
        \end{tabular}
    \end{center}

\section{Probabilities}



\end{document}
